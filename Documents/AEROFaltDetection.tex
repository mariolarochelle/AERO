\documentclass[journal]{IEEEtran}
\usepackage[english]{babel}

\usepackage[utf8]{inputenc}
%\usepackage[spanish]{babel}
\begin{document}

\title{Fault detection methods for tower sensors}

\author{Franco Piergallini Guida, Filippo Visco-Comandini}% <-this % stops a space



% make the title area
\maketitle

\begin{abstract}
The abstract goes here.
\end{abstract}
\section{Introduction}
Renewable energy source is playing an important role in the global energy mix, as a mean of reducing the impact of energy production on climate change. \\
Supervisory control and data acquisition (SCADA) is an application that collects data from a system and sends them to a central computer for monitoring and controlling. Current CM systems essentially provide the necessary sensor and capability of data capture required for monitoring
A wind turbine (WT) is a machine used for converting the kinetic energy in wind into mechanical energy. 
\subsection{Notas bibliografica}
Diagnosis and prognosis of the wind turbine based upon SCADA data using a AI based framework \cite{wang2014scada}.

Fault diagnosis techiques for meteorogial masts \cite{hasu2006weather}  .

Mathematical methods for SCADA data mining of onshore wind farms: Performance evaluation and wakeanalysis \cite{astolfi2016mathematical}.


\cite{kusiak2011prediction}

\cite{lu2009review}

\cite{schlechtingen2012condition}

\cite{schlechtingen2011comparative}

\cite{yang2014wind}
Survey of the wind turbine condition monitoring done in 2014 \cite{tchakoua2014wind}.


\cite{wymore2015survey}

Exploration of exisisting wind turbine SCADA data fro development of fault detections and diagnostic technicaques for wind turbine using clustering algorithms and principal components analysis \cite{kim2011use}.

\section{scada data}
\section{Failure in the sensor towers}
In this section, we are going to describe different existing types of failures in sensor towers. We can classify them in three group
\subsection{Mechanical faults} Mechanicals faults are identify by a mechanical failure in the sensor. For example in the \emph{XXX veleta}, 
in the anemometers .there is a large variety of mechanical faults
\subsection{Connection  faults} Intermittant faults
\subsection{Calibration faults} Faults that changes the measurement all along the scales

\section{Sensor tower diagnostics with scada data}

\section{Algorithms}
\section{Ratio}
\section{Pearson correlations}
\section{Results and discussion}
\section{Conclusion}
XXXXX

\bibliographystyle{IEEEtran}
\bibliography{BibliografiaAERO.bib}

\end{document}


